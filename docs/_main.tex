% Options for packages loaded elsewhere
\PassOptionsToPackage{unicode}{hyperref}
\PassOptionsToPackage{hyphens}{url}
%
\documentclass[
  spanish,
]{article}
\usepackage{amsmath,amssymb}
\usepackage{lmodern}
\usepackage{iftex}
\ifPDFTeX
  \usepackage[T1]{fontenc}
  \usepackage[utf8]{inputenc}
  \usepackage{textcomp} % provide euro and other symbols
\else % if luatex or xetex
  \usepackage{unicode-math}
  \defaultfontfeatures{Scale=MatchLowercase}
  \defaultfontfeatures[\rmfamily]{Ligatures=TeX,Scale=1}
\fi
% Use upquote if available, for straight quotes in verbatim environments
\IfFileExists{upquote.sty}{\usepackage{upquote}}{}
\IfFileExists{microtype.sty}{% use microtype if available
  \usepackage[]{microtype}
  \UseMicrotypeSet[protrusion]{basicmath} % disable protrusion for tt fonts
}{}
\makeatletter
\@ifundefined{KOMAClassName}{% if non-KOMA class
  \IfFileExists{parskip.sty}{%
    \usepackage{parskip}
  }{% else
    \setlength{\parindent}{0pt}
    \setlength{\parskip}{6pt plus 2pt minus 1pt}}
}{% if KOMA class
  \KOMAoptions{parskip=half}}
\makeatother
\usepackage{xcolor}
\IfFileExists{xurl.sty}{\usepackage{xurl}}{} % add URL line breaks if available
\IfFileExists{bookmark.sty}{\usepackage{bookmark}}{\usepackage{hyperref}}
\hypersetup{
  pdftitle={Instalación de R y RStudio},
  pdfauthor={Eva María Mazcuñán Navarro},
  pdflang={es},
  hidelinks,
  pdfcreator={LaTeX via pandoc}}
\urlstyle{same} % disable monospaced font for URLs
\usepackage{longtable,booktabs,array}
\usepackage{calc} % for calculating minipage widths
% Correct order of tables after \paragraph or \subparagraph
\usepackage{etoolbox}
\makeatletter
\patchcmd\longtable{\par}{\if@noskipsec\mbox{}\fi\par}{}{}
\makeatother
% Allow footnotes in longtable head/foot
\IfFileExists{footnotehyper.sty}{\usepackage{footnotehyper}}{\usepackage{footnote}}
\makesavenoteenv{longtable}
\usepackage{graphicx}
\makeatletter
\def\maxwidth{\ifdim\Gin@nat@width>\linewidth\linewidth\else\Gin@nat@width\fi}
\def\maxheight{\ifdim\Gin@nat@height>\textheight\textheight\else\Gin@nat@height\fi}
\makeatother
% Scale images if necessary, so that they will not overflow the page
% margins by default, and it is still possible to overwrite the defaults
% using explicit options in \includegraphics[width, height, ...]{}
\setkeys{Gin}{width=\maxwidth,height=\maxheight,keepaspectratio}
% Set default figure placement to htbp
\makeatletter
\def\fps@figure{htbp}
\makeatother
\setlength{\emergencystretch}{3em} % prevent overfull lines
\providecommand{\tightlist}{%
  \setlength{\itemsep}{0pt}\setlength{\parskip}{0pt}}
\setcounter{secnumdepth}{5}
\usepackage{booktabs}
\ifXeTeX
  % Load polyglossia as late as possible: uses bidi with RTL langages (e.g. Hebrew, Arabic)
  \usepackage{polyglossia}
  \setmainlanguage[]{spanish}
\else
  \usepackage[main=spanish]{babel}
% get rid of language-specific shorthands (see #6817):
\let\LanguageShortHands\languageshorthands
\def\languageshorthands#1{}
\fi
\ifLuaTeX
  \usepackage{selnolig}  % disable illegal ligatures
\fi
\usepackage[]{natbib}
\bibliographystyle{apalike}

\title{Instalación de R y RStudio}
\usepackage{etoolbox}
\makeatletter
\providecommand{\subtitle}[1]{% add subtitle to \maketitle
  \apptocmd{\@title}{\par {\large #1 \par}}{}{}
}
\makeatother
\subtitle{Métodos Numéricos y Estadísticos}
\author{Eva María Mazcuñán Navarro}
\date{}

\begin{document}
\maketitle

{
\setcounter{tocdepth}{2}
\tableofcontents
}
\hypertarget{section}{%
\section*{}\label{section}}

Utilizaremos el lenguaje \textbf{R}, vía la IDE \textbf{RStudio}, para resolver las tareas que se plantearán en la asignatura.

En este documento encontrarás las instrucciones para instalar R y RStudio en tu equipo.

Utiliza la tabla de contenidos del panel izquierdo para elegir el sistema operativo donde quieres realizar la instalación, y sigue los pasos que se describen allí.

\hypertarget{linux}{%
\section{Linux}\label{linux}}

\begin{center}\includegraphics[width=0.15\linewidth]{images/os/tux-flat} \end{center}

\hypertarget{instalaciuxf3n-de-r}{%
\subsection{Instalación de R}\label{instalaciuxf3n-de-r}}

En este apartado se describen las instrucciones para instalar R en Ubuntu. Para otras distribuciones de Linux consulta \href{https://ftp.cixug.es/CRAN/bin/linux/}{este link}.

\begin{center}\includegraphics[width=0.15\linewidth]{images/os/ubuntu} \end{center}

Para instalar R en Ubuntu instala los paquetes \texttt{r-base} y \texttt{r-base-dev}, ejecutando desde la terminal

\begin{verbatim}
sudo apt update
sudo apt install r-base r-base-dev 
\end{verbatim}

También puedes hacer la instalación desde \textbf{Ubuntu Software}.

\hypertarget{instalaciuxf3n-de-rstudio}{%
\subsection{Instalación de RStudio}\label{instalaciuxf3n-de-rstudio}}

Una vez instalado R, sigue los siguientes pasos para instalar RStudio.

\textbf{PASO 1: }
Accede a la página de descargas de RStudio pinchando \href{https://rstudio.com/products/rstudio/download/\#download}{aquí}.

En el apartado \textbf{All Installers} descarga el instalador más reciente para tu versión (para Ubuntu 20, descarga Ubuntu 18/Debian 10).

\textbf{PASO 2: }
Cuando se complete la descarga del paso anterior, haz doble click en el archivo .deb descargado. Se abrirá con el gestor de software de Ubuntu. Presiona el botón `Instalar' para terminar.

\hypertarget{windows}{%
\section{Windows}\label{windows}}

\begin{center}\includegraphics[width=0.15\linewidth]{images/os/windows} \end{center}

A continuación se proporcionan los links para la descarga de los instaladores de R y RStudio en Windows. En ambos casos, se descargará un archivo ejecutable que gestionará la instalación de R / RStudio en tu equipo. Para ejecutarlo, se recomienda hacer click derecho en el archivo y seleccionar `Ejecutar como administrador'. Puedes conservar todas las opciones de instalación que aparecen por defecto.

\hypertarget{instalaciuxf3n-de-r-1}{%
\subsection{Instalación de R}\label{instalaciuxf3n-de-r-1}}

Accede a la página de descargas de R para Windows pinchando \href{https://ftp.cixug.es/CRAN/bin/windows/base/}{aquí}, y haz click el link de título `Download R 4.0.4 for Windows', que encontrarás al principio de la página.

\hypertarget{instalaciuxf3n-de-rstudio-1}{%
\subsection{Instalación de RStudio}\label{instalaciuxf3n-de-rstudio-1}}

Accede a la página de descargas de RStudio pinchando \href{https://rstudio.com/products/rstudio/download/\#download}{aquí}, y, en la tabla del apartado \textbf{All Installers}, haz click en el link de descarga correspondiente a Windows 10/8/7.

\hypertarget{mac}{%
\section{Mac OS X}\label{mac}}

\begin{center}\includegraphics[width=0.15\linewidth]{images/os/apple} \end{center}

\hypertarget{instalaciuxf3n-de-r-2}{%
\subsection{Instalación de R}\label{instalaciuxf3n-de-r-2}}

Accede a la página de descargas de R para Mac OS X pinchando \href{https://ftp.cixug.es/CRAN/bin/macosx/}{aquí} y descarga el instalador correspondiente a la versión de tu equipo.

Solo tienes que ejecutar el archivo pgk que se descargará para instalar R en tu equipo. Puedes conservar todas las opciones de instalación por defecto.

\hypertarget{instalaciuxf3n-de-rstudio-2}{%
\subsection{Instalación de RStudio}\label{instalaciuxf3n-de-rstudio-2}}

Accede a la página de descargas de RStudio pinchando \href{https://rstudio.com/products/rstudio/download/\#download}{aquí}, y, en la tabla del apartado \textbf{All Installers}, haz click en el link de descarga correspondiente a macOS 10.13+.

Se descargará un archivo dmg. Para completar la instalación de RStudio en tu equipo solo tienes que arrastar el icono de la aplicación a tu carpeta Aplicaciones.

  \bibliography{book.bib,packages.bib}

\end{document}
