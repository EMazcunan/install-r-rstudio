\documentclass[
  degree=mecinf,
  title=normal,
  toc=normal,
  bib=normal]{mnye}


\usepackage{booktabs}





\author{%
        Eva María Mazcuñán Navarro
    }


\title{Instalación de R y RStudio}

\term{2020-2021}

\begin{document}

% Before Body

\hypertarget{section}{%
\section*{}\label{section}}

Utilizaremos el lenguaje \textbf{R}, vía la IDE \textbf{RStudio}, para resolver las tareas que se plantearán en la asignatura.

En este documento encontrarás las instrucciones para instalar R y RStudio en tu equipo.

Utiliza la tabla de contenidos del panel izquierdo para elegir el sistema operativo donde quieres realizar la instalación, y sigue los pasos que se describen allí.

\hypertarget{linux}{%
\section{Linux}\label{linux}}

\begin{center}\includegraphics[width=0.15\linewidth]{images/os/tux-flat} \end{center}

\hypertarget{instalaciuxf3n-de-r}{%
\subsection{Instalación de R}\label{instalaciuxf3n-de-r}}

En este apartado se describen las instrucciones para instalar R en Ubuntu. Para otras distribuciones de Linux consulta \href{https://ftp.cixug.es/CRAN/bin/linux/}{este link}.

\begin{center}\includegraphics[width=0.15\linewidth]{images/os/ubuntu} \end{center}

Para instalar R en Ubuntu instala los paquetes \texttt{r-base} y \texttt{r-base-dev}, ejecutando desde la terminal

\begin{verbatim}
sudo apt update
sudo apt install r-base r-base-dev 
\end{verbatim}

También puedes hacer la instalación desde \textbf{Ubuntu Software}.

\hypertarget{instalaciuxf3n-de-rstudio}{%
\subsection{Instalación de RStudio}\label{instalaciuxf3n-de-rstudio}}

Una vez instalado R, sigue los siguientes pasos para instalar RStudio.

\textbf{PASO 1: }
Accede a la página de descargas de RStudio pinchando \href{https://rstudio.com/products/rstudio/download/\#download}{aquí}.

En el apartado \textbf{All Installers} descarga el instalador más reciente para tu versión (para Ubuntu 20, descarga Ubuntu 18/Debian 10).

\textbf{PASO 2: }
Cuando se complete la descarga del paso anterior, haz doble click en el archivo .deb descargado. Se abrirá con el gestor de software de Ubuntu. Presiona el botón `Instalar' para terminar.

\hypertarget{windows}{%
\section{Windows}\label{windows}}

\begin{center}\includegraphics[width=0.15\linewidth]{images/os/windows} \end{center}

A continuación se proporcionan los links para la descarga de los instaladores de R y RStudio en Windows. En ambos casos, se descargará un archivo ejecutable que gestionará la instalación de R / RStudio en tu equipo. Para ejecutarlo, se recomienda hacer click derecho en el archivo y seleccionar `Ejecutar como administrador'. Puedes conservar todas las opciones de instalación que aparecen por defecto.

\hypertarget{instalaciuxf3n-de-r-1}{%
\subsection{Instalación de R}\label{instalaciuxf3n-de-r-1}}

Accede a la página de descargas de R para Windows pinchando \href{https://ftp.cixug.es/CRAN/bin/windows/base/}{aquí}, y haz click el link de título `Download R 4.0.4 for Windows', que encontrarás al principio de la página.

\hypertarget{instalaciuxf3n-de-rstudio-1}{%
\subsection{Instalación de RStudio}\label{instalaciuxf3n-de-rstudio-1}}

Accede a la página de descargas de RStudio pinchando \href{https://rstudio.com/products/rstudio/download/\#download}{aquí}, y, en la tabla del apartado \textbf{All Installers}, haz click en el link de descarga correspondiente a Windows 10/8/7.

\hypertarget{mac}{%
\section{Mac OS X}\label{mac}}

\begin{center}\includegraphics[width=0.15\linewidth]{images/os/apple} \end{center}

\hypertarget{instalaciuxf3n-de-r-2}{%
\subsection{Instalación de R}\label{instalaciuxf3n-de-r-2}}

Accede a la página de descargas de R para Mac OS X pinchando \href{https://ftp.cixug.es/CRAN/bin/macosx/}{aquí} y descarga el instalador correspondiente a la versión de tu equipo.

Solo tienes que ejecutar el archivo pgk que se descargará para instalar R en tu equipo. Puedes conservar todas las opciones de instalación por defecto.

\hypertarget{instalaciuxf3n-de-rstudio-2}{%
\subsection{Instalación de RStudio}\label{instalaciuxf3n-de-rstudio-2}}

Accede a la página de descargas de RStudio pinchando \href{https://rstudio.com/products/rstudio/download/\#download}{aquí}, y, en la tabla del apartado \textbf{All Installers}, haz click en el link de descarga correspondiente a macOS 10.13+.

Se descargará un archivo dmg. Para completar la instalación de RStudio en tu equipo solo tienes que arrastar el icono de la aplicación a tu carpeta Aplicaciones.

% After Body

\end{document}
